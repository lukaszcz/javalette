% -*- coding: utf-8 -*-
\documentclass[a4paper,12pt]{article}
\usepackage[polish]{babel}
\usepackage[OT4]{fontenc}
\usepackage[utf8]{inputenc}
\usepackage{amsfonts}
\usepackage{algorithmic}
\usepackage{algorithm}
\usepackage{longtable}
\usepackage{graphicx}
\usepackage{color}
\usepackage[colorlinks=true]{hyperref}

\floatname{algorithm}{Algorytm}
\newtheorem{twierdzenie}{Twierdzenie}
\newtheorem{definicja}{Definicja}
\newtheorem{fakt}{Fakt}
\newtheorem{lemat}{Lemat}


\begin{document}

\title{Kompilator Javalette}
\author{Łukasz Czajka}
\date{\today}
\maketitle

\newpage

\tableofcontents

\newpage

\section{Frontend}

Parser zaimplementowany w pliku {\tt parse.y} generuje drzewo składniowe
używając funkcji {\tt new\_node()} z pliku {\tt tree.h}. Moduł {\tt
tree.c} jest odpowiedzialny za sprawdzenie poprawności kontekstowej tego
drzewa, oraz częściowo za generację z niego kodu pośredniego poprzez
wywoływanie odpowiednich funkcji z modułu {\tt quadr.h}. Plik {\tt
types.c} odpowiada za (bardzo prostą) analizę równoważności typów.

\section{Kod pośredni}

Struktury kodu pośredniego zdefiniowane są w pliku {\tt quadr.h}, a
operacje na nich w {\tt quadr.c}.

Kod pośredni jest rodzjem kodu czwórkowego. Różni się tym od tego dla
interpretera {\em iquadr}, że nie operuje na rejestrach, tylko na
zmiennych, które są lokalne dla jednego wywołania funkcji.

Kod pośredni generowany jest od razu z podziałem na bloki bazowe, a
etykiety utożsamiane są z początkami tych bloków.

Nie są rozróżniane zmienne tymczasowe od nietymczasowych. Generator kodu
tego nie potrzebuje. Potrzebuje za to globalną informację o żywotności
zmiennych.

%Moduł {\tt quadr.c} generuje kod pośredni w standardowy sposób

\section{Analiza żywotności}

Globalna analiza żywotności zmiennych zaimplementowana jest w pliku {\tt
flow.c} w funkcji {\tt analyze\_liveness()}. Funkcja ta zakłada, że
wcześniej wywołano {\tt create\_block\_graph()} w celu stworzenia grafu
przepływu z bloków bazowych. Ze względu na postać kodu pośredniego każdy
wierzchołek w tym grafie może mieć co najwyżej dwa następniki.

Dla każdego bloku obliczamy zmienne żywe na początku bloku ({\tt
vars\_at\_start} w {\tt basic\_block\_t}) oraz na końcu ({\tt
live\_at\_end}).

Algorytm analizy żywotności można by zakwalifikować jako nietrywialny,
lecz jest on zupełnie standardowy, więc nie będę tutaj przepisywać
książki. Nadmienię tylko, że wszystkie zbiory (in, out, gen, kill)
zaimplementowałem jako drzewa czerwono-czarne.

Faza analizy żywotności oblicza także dla każdej zmiennej żywej na
początku bloku odległość jej najbliższego użycia (NUD), definiowaną jako
ilość instrukcji od początku bloku do pierwszego użycia tej zmiennej w
bloku, lub minimum z NUD dla następników tego bloku w grafie przepływu.

\section{Generacja kodu wynikowego}

Kod wynikowy generowany jest z kodu pośredniego poprzez moduł {\tt
gencode.c} w połączeniu z odpowiednim obiektem {\tt backend\_t}, który
udostępnia funkcje wirtualne\footnote{No tak ściśle rzecz biorąc, to
jest to struktura ze wskaźnikami do funkcji, bo pisałem to w C. Ale
przecież liczą się idee a nie nazwy\ldots} do wypisywania kodu dla
odpowiednich konstrukcji. Moduł {\tt gencode.c} bezpośrednio zajmuje się
tylko alokacją rejestrów, śledzeniem ich zawartości oraz decydowaniem co
kiedy przesłać do pamięci lub zachować na boku w innym rejestrze, a przy
tym także analizą żywotności na poziomie jednego bloku bazowego
(korzystając z informacji globalnych).

Standardowa metoda generacji kodu poprzez śledzenie zawartości rejestrów
jest tutaj rozszerzona w ten sposób, że na końcu bloku {\em nie}
odsyłamy zmiennych żywych do pamięci. Oczywiście trzeba jednak zapewnić,
aby następnicy tego bloku w grafie przepływu wiedzieli, gdzie jest
zmienna.

Sposób generacji kodu pośredniego oraz kolejność generacji kodu
wynikowego dla bloków bazowych zapewniają, że w momencie generowania
kodu dla bloku był już generowany kod dla co najwyzej jednego jego
następnika. Ponadto każda etykieta (utożsamiana z początkiem jakiegoś
bloku bazowego) jest celem co najwyżej jednego skoku. Można to zapewnić
dodając zawsze odpowiednią liczbę pustych bloków. Te obserwacje
uzasadniają poprawność poniższego algorytmu.

Zmiennym żywym na początku bloku, dla którego jeszcze nie generowaliśmy
kodu, przypisywać będziemy lokacje na podstawie ich lokacji na końcu
bloków-poprzedników w grafie przepływu.

Dla każdej zmiennej $v$ żywej na końcu aktualnego bloku $B$, sprawdź czy
w którymś z następników $B$ ta zmienna ma już przypisaną lokację. Ze
względu na powyższe obserwacje może być co najwyżej jeden taki
następnik. Jeśli nie, to przypisz obecne lokacje zmiennej do obu
następników. Jeśli tak, to przenieś zmienną do jakiejś lokacji z tego
następnika, oraz usuń pozostałe lokacje jeśli kod dla tego następnika
nie był jeszcze generowany, lub przenieś zmienną do wszystkich lokacji,
jeśli był. Nie jest wykluczone, że obecne lokacje i lokacje w następniku
się pokrywają -- wtedy jeśli kod dla następnika nie był generowany, to
wystarczy usunąć te, które się nie powtarzają. Następnie uaktualniamy
lokacje tej zmiennej w drugim następniku.

Może się potem okazać, że w pustych blokach, które dodaliśmy na etapie
generacji kodu pośredniego trzeba będzie wygenerować jakieś przypisania.

Powyższa metoda nieco skomplikowała algorytm generacji kodu, ale pozwala
na generowanie całkiem dobrego kodu, pod względem odwołań do pamięci,
nawet bez globalnej alokacji rejestrów.


\subsection{Alokacja rejestrów}

Alokacja rejestrów zaimplementowana jest w funkcji {\tt bellady\_ra} w
pliku {\tt gencode.c}. Jest to strategia Bellady'ego wyboru rejestru dla
którego średnia najbliższego użycia dla przechowywanych w nim wartości
jest najmniejsza spośród rejestrów, których wybór spowoduje odesłanie do
pamięci najmniejszej liczby zmiennych.

\subsection{Dla interpretera iquadr}

Generacja kodu dla interpretera {\em iquadr} zaimplementowana jest w
pliku {\tt quadr\_backend.c}. Implementuje ona po prostu funkcje z {\tt
backend\_t}.

\subsection{Na intela i386}

Dla kodu assemblera NASM jest podobnie jak w powyższym punkcie (plik
{\tt i386\_backend.c}) z tym, że implementacja tych funkcji jest już
nieco mniej trywialna.

\section{Optymalizacje}

\subsection{Lokalna optymalizacja bloków bazowych}

Optymalizacja bloków bazowych zaimplementowana jest w pliku {\tt
opt.cpp}. Obejmuje ona zwijanie stałych, eliminację podwyrażeń wspólnych
oraz propagację kopii. Wykorzystywany jest standardowy algorytm budowy
DAGa dla bloku bazowego, który można znaleźć np. w \cite{kompilatory}.

\subsection{Pomijanie wskaźnika ramki}

Pomijany jest wskaźnik ramki stosu. Adresowanie odbywa się od czubka
stosu. Maksymalna wysokość stosu dla jednej funkcji jest wyliczana w
trakcie generacji kodu, po czym funkcja {\tt fix\_stack()} z pliku {\tt
outbuf.c} poprawia odwołania do stosu uwzględniając tę informację.

\section{Tablice}

Zaimplementowałem jakąś podstawową wersję tablic. Nie mogą być one
kopiowane ani przekazywane jako parametr. Mają stałą wielkość i nie mogą
być wielowymiarowe. Kod dla nich generowany nie jest specjalnie
optymalny.

\section{Testy}

\begin{itemize}
 \item Tablice: {\tt good004.jl}, {\tt bad104.jl}, {\tt bad105.jl}, {\tt
       bad106.jl}.
 \item Optymalizacje: {\tt good006.jl}, {\tt good007.jl}. Można zobaczyć
       kod wyniowy. Najlepiej wychodzi dla interpretera {\em iquadr}.
 \item Pozostałe ,,feature'y'' są raczej integralną częścią kompilatora,
       bez których działać on nie potrafi. W związku z tym ich działanie
       można zaobserwować na dowolnym z testów.
\end{itemize}

\section{Znane niedociągnięcia}

\begin{itemize}
 \item Nie zdążyłem zaimplementować optymalizacji przez szparkę dla
       assemblera x86, więc mimo włączonych optymalizacji pojawia się
       trochę ,,głupich'' instrukcji w stylu {\tt sub esp,0} czy {\tt
       fadd st1; fstp st0}, które mogłyby być w prosty sposób ulepszone.
 \item Przekazywanie parametrów w rejestrach też nie działa mimo, że
       jest taka opcja. Nie zdążyłem zdebugować.
 \item Opcja {\tt -O2} nie działa.
 \item Używanie tablic pogarsza efekty optymalizacji bloków bazowych.
\end{itemize}

\begin{thebibliography}{00}
 \bibitem{kompilatory} Aho, Sethi, Ullman, {\it Kompilatory}.
\end{thebibliography}

\end{document}


